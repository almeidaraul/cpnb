\documentclass[a4paper]{article}
\usepackage[margin=1cm,bmargin=2cm,footskip=1cm]{geometry}
\usepackage{listings}
\usepackage[T1]{fontenc}
\usepackage{multicol}
\usepackage[utf8]{inputenc} % TODO novo
%\usepackage[brazilian]{babel}
\usepackage{lmodern}
\usepackage{amsmath}
\usepackage{caption}
\usepackage{multirow}
\usepackage{fontspec}
\usepackage{hyperref}
\usepackage{xcolor}
\usepackage{amssymb}
\usepackage{colortbl}
\setmonofont[
  Path = TTF/,
  UprightFont = *-Regular,
  BoldFont = *-Bold,
  ItalicFont = *-Italic,
  BoldItalicFont = *-BoldItalic,
  Extension = .ttf
]{Fantasque Sans Mono}

\definecolor{commentgreen}{RGB}{2,112,10}
\definecolor{eminence}{RGB}{210,31,60}

\newcommand{\bigO}{\mathcal{O}}
\lstset{
  numbers=left,
  breaklines=true,
  breakatwhitespace=true,
  numbersep=5pt,
  %rulecolor=\color{black},
  xleftmargin=\parindent,
  basicstyle=\footnotesize\ttfamily,
  showspaces=false,
  showstringspaces=false,
  language=C++,
  columns=fullflexible,
  commentstyle=\color{commentgreen},
  keywordstyle=\color{eminence},
  escapeinside={<@}{@>},
  tabsize=2,
}
\lstset{literate=
  {á}{{\'a\lst@whitespacefalse}}1 {é}{{\'e\lst@whitespacefalse}}1
  {í}{{\'i\lst@whitespacefalse}}1 {ó}{{\'o\lst@whitespacefalse}}1
  {ú}{{\'u\lst@whitespacefalse}}1
  {Á}{{\'A\lst@whitespacefalse}}1 {É}{{\'E\lst@whitespacefalse}}1
  {Í}{{\'I\lst@whitespacefalse}}1 {Ó}{{\'O\lst@whitespacefalse}}1
  {Ú}{{\'U\lst@whitespacefalse}}1
  {à}{{\`a\lst@whitespacefalse}}1 {è}{{\`e\lst@whitespacefalse}}1
  {ì}{{\`i\lst@whitespacefalse}}1 {ò}{{\`o\lst@whitespacefalse}}1
  {ù}{{\`u\lst@whitespacefalse}}1
  {À}{{\`A\lst@whitespacefalse}}1
  {È}{{\'E\lst@whitespacefalse}}1 {Ì}{{\`I\lst@whitespacefalse}}1
  {Ò}{{\`O\lst@whitespacefalse}}1 {Ù}{{\`U\lst@whitespacefalse}}1
  {ä}{{\"a\lst@whitespacefalse}}1 {ë}{{\"e\lst@whitespacefalse}}1
  {ï}{{\"i\lst@whitespacefalse}}1 {ö}{{\"o\lst@whitespacefalse}}1
  {ü}{{\"u\lst@whitespacefalse}}1
  {Ä}{{\"A\lst@whitespacefalse}}1 {Ë}{{\"E\lst@whitespacefalse}}1
  {Ï}{{\"I\lst@whitespacefalse}}1 {Ö}{{\"O\lst@whitespacefalse}}1
  {Ü}{{\"U\lst@whitespacefalse}}1
  {â}{{\^a\lst@whitespacefalse}}1 {ê}{{\^e\lst@whitespacefalse}}1
  {î}{{\^i\lst@whitespacefalse}}1 {ô}{{\^o\lst@whitespacefalse}}1
  {û}{{\^u\lst@whitespacefalse}}1
  {Â}{{\^A\lst@whitespacefalse}}1 {Ê}{{\^E\lst@whitespacefalse}}1
  {Î}{{\^I\lst@whitespacefalse}}1 {Ô}{{\^O\lst@whitespacefalse}}1
  {Û}{{\^U\lst@whitespacefalse}}1
  {Ã}{{\~A\lst@whitespacefalse}}1 {ã}{{\~a\lst@whitespacefalse}}1
  {Õ}{{\~O\lst@whitespacefalse}}1 {õ}{{\~o\lst@whitespacefalse}}1
  {œ}{{\oe\lst@whitespacefalse}}1 {Œ}{{\OE\lst@whitespacefalse}}1
  {æ}{{\ae\lst@whitespacefalse}}1 {Æ}{{\AE\lst@whitespacefalse}}1
  {ß}{{\ss\lst@whitespacefalse}}1
  {ű}{{\H{u}\lst@whitespacefalse}}1 {Ű}{{\H{U}\lst@whitespacefalse}}1
  {ő}{{\H{o}\lst@whitespacefalse}}1 {Ő}{{\H{O}\lst@whitespacefalse}}1
  {ç}{{\c c\lst@whitespacefalse}}1 {Ç}{{\c C\lst@whitespacefalse}}1
  {ø}{{\o\lst@whitespacefalse}}1
  {å}{{\r a\lst@whitespacefalse}}1 {Å}{{\r A\lst@whitespacefalse}}1
  {€}{{\euro\lst@whitespacefalse}}1 {£}{{\pounds\lst@whitespacefalse}}1
  {«}{{\guillemotleft\lst@whitespacefalse}}1
  {»}{{\guillemotright\lst@whitespacefalse}}1
  {ñ}{{\~n\lst@whitespacefalse}}1 {Ñ}{{\~N\lst@whitespacefalse}}1
  {¿}{{?`\lst@whitespacefalse}}1
  {º}{{\textordmasculine\lst@whitespacefalse}}1
  %{~}{{\raisebox{0.5ex}{\texttildelow}\lst@whitespacefalse}}1
  %{^}{{\raisebox{-0.75ex}{\^{}}\lst@whitespacefalse}}1
}

\title{Competitive Programming Notebook}
\author{Raul Almeida$^1$}
\date{$^1$Universidade Federal do Paraná\\[2ex]\today}

\begin{document}
\twocolumn
\maketitle

% \maketitle
\tableofcontents
\begin{table*}
\centering
  \begin{tabular}{lll}
    \hline
    $n$ & not-TLE algorithm & Example \\ \hline
    $\leq [10..11]$ & $\bigO(n!)$, $\bigO(n^6)$ & Enumerate permutations \\
    $\leq [15..18]$ & $\bigO(2^n n^2)$ & TSP with DP \\
    $\leq [18..22]$ & $\bigO(2^n n)$ & Bitmask DP \\
    $\leq 100$ & $\bigO(n^4)$ & 3D DP with $\bigO(n)$ loop \\
    $\leq 400$ & $\bigO(n^3)$ & Floyd-Warshall \\
    $\leq 2\cdot 10^3$ & $\bigO(n^2 \lg n)$ & 2 nested loops + tree query \\
    $\leq 5\cdot 10^4$ & $\bigO(n^2)$ & Bubble/Selection/Insertion Sort \\
    $\leq 10^5$ & $\bigO(n \lg^2 n) = \bigO((\lg n)(\lg n))$ & Build suffix array \\
    $\leq 10^6$ & $\bigO(n \lg n)$ & MergeSort, build SegTree \\
    $\leq 10^7$ & $\bigO(n \lg \lg n)$ & Sieve, totient function \\
    $\leq 10^8$ & $\bigO(n)$, $\bigO(\lg n)$, $\bigO(1)$ & Mathy solution often with IO bottleneck ($n \leq 10^9$) \\
    \hline
  \end{tabular}
  \caption*{$10^8$ ops/second}
\end{table*}


\newpage
\section{Theory}
\subsection{Relevant comparisons}
\begin{table}[ht]
  \centering
  \begin{tabular}{l|r}
    \hline
    $\lg 10$ (\texttt{1E1}) & 2.3 \\
    $\lg 100$ (\texttt{1E1}) & 4.6 \\
    $\lg 1000$ (\texttt{1E2}) & 6.9 \\
    $\lg 10000$ (\texttt{1E3}) & 9.2 \\
    $\lg 100000$ (\texttt{1E4}) & 11.5 \\
    $\lg 1000000$ (\texttt{1E5}) & 13.8 \\
    $\lg 10000000$ (\texttt{1E6}) & 16.1 \\
    $\lg 100000000$ (\texttt{1E7}) & 18.4 \\
    $\lg 1000000000$ (\texttt{1E8}) & 20.7 \\
    $\lg 10000000000$ (\texttt{1E9}) & 23.0 \\
    $\lg 100000000000$ (\texttt{1E10}) & 25.3 \\
    $\lg 1000000000000$ (\texttt{1E11}) & 27.6 \\
    $\lg 10000000000000$ (\texttt{1E12}) & 29.9 \\
    $2^{10}$ & $\approx 10^{3}$ \\
    $2^{20}$ & $\approx 10^{6}$ \\
    \hline
  \end{tabular}
\end{table}

\begin{table}[ht]
  \centering
  \begin{tabular}{crcccc}
    \hline
    Sign & Type & Bits & Max & Digits \\
    \hline
    $\pm$ & \texttt{char} & 8 & 127 & 2 \\
     $+$  & \texttt{unsigned char} & 8 & 255 & 2 \\
    $\pm$ & \texttt{short} & 16 & 32\,767 & 4 \\
     $+$  & \texttt{unsigned short} & 16 & 65\,535 & 4 \\
    $\pm$ & \texttt{int}/\texttt{long} & 32 & $\approx 2 \cdot 10^{9}$ & 9\\
     $+$  & \texttt{unsigned int}/\texttt{long} & 32 & $\approx 4 \cdot 10^{9}$ & 9 \\
    $\pm$ & \texttt{long long} & 64 & $\approx 9 \cdot 10^{18}$ & 18 \\
     $+$  & \texttt{unsigned long long} & 64 & $\approx 18 \cdot 10^{18}$ & 19 \\
    $\pm$ & \texttt{\_\_int128} & 128 & $\approx 17 \cdot 10^{37}$ & 38 \\
     $+$  & \texttt{unsigned \_\_int128} & 128 & $\approx 3 \cdot 10^{38}$ & 38 \\
    \hline
  \end{tabular}
\end{table}

\begin{table}[ht]
  \centering
  \begin{tabular}{lcc}
    \hline
    Algorithm & Time & Space \\
    \hline
    %aqui
    \hline
    ArticBridges & $\mathcal{O}(V+E)$ & $\mathcal{O}(V+E)$ \\
    Bellman-Ford & $\mathcal{O}(VE)$ & $\mathcal{O}(V+E)$ \\
    Dijksta & $\mathcal{O}((V+E) \log V)$ & $\mathcal{O}(V^2)$ \\
    Edmond Karp & $\mathcal{O}(VE^2)$ & $\mathcal{O}(V+E)$ \\
    Euler Tour & $\mathcal{O}(E^2)$ & \\
    Floyd Warshall & $\mathcal{O}(V^3+E)$ & $\mathcal{O}(V^2+E)$ \\
    Graph Check & $\mathcal{O}(V+E)$ & $\mathcal{O}(V+E)$ \\
    Kahn & $\mathcal{O}(VE)$ & $\mathcal{O}(V+E)$ \\
    Kruskal & $\mathcal{O}(E \log V)$ & $\mathcal{O}(V+E)$ \\
    LCA & $\mathcal{O}(N \log N)$ & $\mathcal{O}(N \log N)$ \\
    MCBM & $\mathcal{O}(VE)$ & \\
    Prim & $\mathcal{O}(E \log V)$ & $\mathcal{O}(V+E)$ \\
    Tarjan & $\mathcal{O}(V+E)$ & $\mathcal{O}(V+E)$ \\
    Extended Euclid & $\mathcal{O}(\log \min(a, b))$ & $\mathcal{O}(1)$ \\
    Floyd (cycle) & $\mathcal{O}(V)$ & $\mathcal{O}(1)$ \\
    PrimeFac + OptTrialDiv & $\mathcal{O}(\pi(\sqrt{n}))$ & $\mathcal{O}(n)$ \\
    Sieve of Eratosthenes & $\mathcal{O}(n \log \log n)$ & $\mathcal{O}(n)$ \\
    Binary Search & $\mathcal{O}(\log N)$ & \\
    Coordinate Compression & $\mathcal{O}(N \log N)$ & \\
    KMP & $\mathcal{O}(N)$ & \\
    MUF & $\mathcal{O}(AM)$ & $\mathcal{O}(N)$ \\
    Bottom-Up SegTree & $\mathcal{O}(\log N)$ & $\mathcal{O}(N)$ \\
    \hline
  \end{tabular}
\end{table}

\subsection{Prime counting function - pi(x)}
Asymptotic to $\frac{x}{\log x}$ by the prime number theorem.

\begin{table}[ht]
  \centering
  \begin{tabular}{|c|c|c|c|c|c|c|c|c|}
  \hline
    \rowcolor{gray!40}
    x&10&$10^2$&$10^3$&$10^4$\\ \hline
    $\pi(x)$& 4 & 25 & 168 & 1\,229 \\ \hline
    \rowcolor{gray!40}
    x&$10^5$&$10^6$&$10^7$&$10^8$\\ \hline
    $\pi(x)$& 9\,592 & 78\,498 & 664\,579 & 5\,761\,455\\ \hline
  \end{tabular}
\end{table}


\subsection{Progressions}
\begin{align*}
  a_n = a_k + r(n - k) \\
  a_n = a_k q^{(n-k)}
\end{align*}
\begin{itemize}
  \item $r$, $q$: Ratio
  \item $k$: Known term
  \item $n$: Term you want
\end{itemize}
\begin{align*}
  S_n = \frac{n(a_1 + a_n)}{2} \\
  S_n = \frac{a_1(q^n - 1)}{q-1}
\end{align*}

\subsection{Series Identities}
$$\sum_{i=1}^{n} i^{2} = \frac{n(n+1)(2n+1)}{6}  \qquad  \sum_{i=1}^{n} i^{3} = \frac{n^{2}(n+1)^{2}}{4} = \left(\sum_{i=1}^n i\right)^2$$

$$ g_k(n) = \sum_{i=1}^n i^k = \frac{1}{k+1} \left( n^{k+1} + \sum_{j=1}^k \binom{k+1}{j+1} (-1)^{j+1} g_{k-j}(n) \right) $$

$$\sum_{i=0}^{n} ic^{i} = \frac{nc^{n+2} - (n+1)c^{n+1} + c}{(c-1)^{2}}, \quad c \neq 1$$

$$\sum_{i=0}^{\infty} ic^{i} = \frac{c}{(1-c)^{2}}, \quad |c| < 1$$

$$l + (l+1) + \dots + r = \frac{(l+r)\cdot(r-l+1)}{2}$$

\subsection{Binomial Identities}

$$
\binom{n}{k} = \frac{n}{k}\binom{n-1}{k-1}
$$
$$
\binom{n-1}{k} - \binom{n-1}{k-1} = \frac{n - 2k}{k} \binom{n}{k}
$$
$$
\binom{n}{h}\binom{n-h}{k} = \binom{n}{k}\binom{n-k}{h}
$$
$$
\binom{n}{k} = \frac{n+1-k}{k} \binom{n}{k-1}
$$
$$
\sum_{k = 0}^n k\binom{n}{k} = n 2^{n-1}
$$
$$
\sum_{k = 0}^n k^2 \binom{n}{k} = (n + n^2)2^{n-2}
$$
$$
\sum_{j = 0}^k\binom{m}{j} \binom{n-m}{k-j} = \binom{n}{k}
$$
$$
\sum_{j = 0}^m \binom{m}{j}^2 = \binom{2m}{m}
$$
$$
\sum_{m = 0}^n \binom{m}{j} \binom{n-m}{k-j} = \binom{n+1}{k+1}
$$
$$
\sum_{m = k}^n \binom{m}{k} = \binom{n+1}{k+1}
$$
$$
\sum_{r = 0}^m \binom{n+r}{r} = \binom{n+m+1}{m}
$$
$$
\sum_{k=0}^{\lfloor n/2 \rfloor} \binom{n-k}{k} = \text{Fib}(n+1)
$$
$$
(x + y)^{n} = \sum_{k=0}^{n} \binom{n}{k} x^{n-k} y^{k}
$$
$$
(1 + x)^{n} = \sum_{k=0}^{n} \binom{n}{k} x^{k}
$$
$$
2\sum_{i = L}^R \binom{n}{i} - \binom{n}{L} - \binom{n}{R} = \sum_{i = L+1}^R \binom{n+1}{i}
$$

\subsection{Lucas' Theorem}
\begin{align*}
  \binom{n}{m} = \Pi_{i=0}^{k} \binom{n_i}{m_i} \pmod{p}
\end{align*}
For prime $p$, $n_i$ and $m_i$ are coefficients of the representations of $n$ and $m$ in base $p$.

\subsection{Fermat Theorems}
$p$ is prime
\begin{align*}
  a^p = a \pmod{p} \\
  a^{p-1} = 1 \pmod{p} \\
  (a + b)^p = a^p + b^p \pmod{p} \\
  a^{-1} = a^{p-2} \pmod{p}
\end{align*}

\subsection{Modulo @ exponent}
For coprime $a, m$:
$$ a^n \equiv a^{n \text{ mod } \varphi(m)} \quad (\text{mod } m) $$

Generally, if $n \geq \log_2 m$, then

$$ a^n \equiv a^{\varphi(m) + [n \text{ mod } \varphi(m)]} \quad (\text{mod } m) $$

\subsection{Heron's Formula}
Area of a triangle ($s = \frac{a+b+c}{2}$)
\begin{align*}
  A = \sqrt{s(s-a)(s-b)(s-c)}
\end{align*}

\subsection{Some Primes}
\begin{multicols}{2}
\begin{itemize}
  \itemsep -2pt
  \item $10^6 + 69$
  \item $10^9 + 7$
  \item $10^9 + 9$
  \item $10^{18}-11$
  \item $10^{18}+3$
  \item $2^{61}-1$
  \item $1000696969$
  \item $998244353$
  \item 999999937
  \item 1000000007
  \item 1000000009
  \item 1000000021
  \item 1000000033
  \item $10^{18} - 11$
  \item $10^{18} + 3$
  \item $2305843009213693951 = 2^{61} - 1$
\end{itemize}
\end{multicols}

\subsection{Catalan Numbers}
{\sloppy
1, 1, 2, 5, 14, 42, 132, 429, 1430, 4862, 16796, 58786, 208012, 742900, 2674440,
9694845, 35357670, 129644790, 477638700, 1767263190, 6564120420,
24466267020, 91482563640, \\343059613650, 1289904147324, 4861946401452,
\\18367353072152, 69533550916004, 263747951750360, \\1002242216651368.}

\begin{align*}
  C_n = \frac{1}{n+1} \binom{2n}{n} = \frac{(2n)!}{(n+1)!n!} = \Pi^n_{k=2} \frac{n+k}{k}, n \ge 0.
\end{align*}

$C_n$ is:
\begin{itemize}
  \itemsep -2pt
  \item The number of valid parenthesis strings with $n$ parentheses
  \item The number of complete binary trees with $n+1$ leaves
  \item How many times a $n+2$-sided convex polygon can be cut in triangles conecting its vertices with straight lines
\end{itemize}

\subsection{Binomial}
$X$ is the number of successes in a sequence of $n$ independent experiments.  $P(X = k) = \binom{n}{k}\,p^{k}(1-p)^{n-k}$, and $E[X] = np$ and $Var(X) = np(1-p)$.

\subsection{Trigonometry}
$\sin^2\theta + \cos^2\theta = 1$, $\sin = \frac{opo}{hip}$, $\cos = \frac{adj}{hip}$, $\tan = \frac{opo}{adj}$. $\sin \theta = x \to \arcsin x = \theta$.

$\alpha$ degrees to $x$ rd: $\alpha = \frac{180x}{\pi}$

\subsection{Multiples of gcd}
Multiples of $\textrm{gcd}(A, B)$ that are $\in [0, A)$

Let $A, B > 0$, $g = \textrm{GCD}(A, B)$, $A = ag$ and $B = bg$.

$a$ integers
$(0\times B) \% A,\; (1\times B) \% A,\; (2\times B) \% A \dotsc ((a-1)\times B) \% A$
correspond to each multiple of $g$ between $0$ and $A-1$ (inclusive); note that they
are all unique.

\subsection{Mod value range}
$A < B \implies A\%B=A$, $A \geq B \implies A\%B \leq A/2$

\subsection{Expected Value}
Avg value of event. For each event, add to the sum the probability of an event times the value of $X$ in that event
$\mathbb{E}(X) = \sum_{\omega \in \Omega} (P(\omega)\times X(\omega))$

Another way of looking at it:

$\mathbb{E}(X) = \sum_{i=1}^{M} (i \times P(X=i))$

Since in the expanded version of this sum $P(X=i)$ will appear $i$ times, you're also calculating for each $i$ the probability that $X \geq i$ ($P(x=M)$ will appear $M$ times, once for each $i$; $P(x=1)$ will appear exactly once, for $i=1$; and so on). So

$\mathbb{E}(X) = \sum_{i=1}^{M} (i \times P(X=i)) = \sum_{i=1}^{M} P(X \geq i)$

\subsection{Combination}
A combination ${}_nC_k = \binom{n}{k}$ ($n$ \textit{chooses} $k$) refers to selecting $k$ objects from a collection of $n$ where the order of choice doesn't matter.

\textbf{Without repetition:} can't choose an element twice. $\binom{n}{k} = \frac{n!}{r!(n-k)!}$

\textbf{With repetition:} elements may be chosen more than once. $\binom{n}{k} = \frac{(k+n-1)!}{k!(n-1)!}$

\subsection{Permutation}
A permutation ${}_nP_k$ refers to selecting $k$ objects from a collection of $n$ where the order of choice matters.

\textbf{With repetition:} elements may be chosen more than once. ${}_nP_k = n^k$

\textbf{Without repetition:} can't choose an element twice. ${}_nP_k = \frac{n!}{(n-k)!}$

\section{Emergency}
\noindent
\textbf{Pre-submit}

\noindent
Write a few simple test cases if sample is not enough.

\noindent
Are time limits close? If so, generate max cases.

\noindent
Is the memory usage fine?

\noindent
Could anything overflow?

\noindent
Make sure to submit the right file (check the filename you're editing).

\noindent
\textbf{Wrong answer}

\noindent
Print your solution and debug output!

\noindent
Are you clearing all data structures between test cases?

\noindent
Can your algorithm handle the whole range of input?

\noindent
Read the full problem statement again.

\noindent
Do you handle all corner cases correctly?

\noindent
Have you understood the problem correctly?

\noindent
Any uninitialized variables?

\noindent
Any overflows?

\noindent
Confusing \texttt{N} and \texttt{M}, \texttt{i} and \texttt{j}, etc.?

\noindent
Are you sure your algorithm works?

\noindent
What special cases have you not thought of?

\noindent
Are you sure the STL functions you use work as you think?

\noindent
Add some assertions, maybe resubmit.

\noindent
Create some testcases to run your algorithm on.

\noindent
Go through the algorithm for a simple case.

\noindent
Go through this list again.

\noindent
Explain your algorithm to a teammate.

\noindent
Ask the teammate to look at your code.

\noindent
Go for a small walk, e.g. to the toilet.

\noindent
Is your output format correct? (including whitespace)

\noindent
Rewrite your solution from the start or let a teammate do it.

\noindent
\textbf{Runtime error}

\noindent
Have you tested all corner cases locally?

\noindent
Any uninitialized variables?

\noindent
Are you reading or writing outside the range of any vector?

\noindent
Any assertions that might fail?

\noindent
Any possible division by 0? (\texttt{mod 0} for example)

\noindent
Any possible infinite recursion?

\noindent
Invalidated pointers or iterators?

\noindent
Are you using too much memory?

\noindent
Debug with resubmits (e.g. remapped signals, see Various).

\noindent
\textbf{Time limit exceeded}

\noindent
Do you have any possible infinite loops?

\noindent
What is the complexity of your algorithm?

\noindent
Are you copying a lot of unnecessary data? (use references)

\noindent
How big is the input and output? (consider \texttt{scanf} and \texttt{printf})

\noindent
Avoid \texttt{vector}, \texttt{map}. (use \texttt{array}/\texttt{unordered\_map})

\noindent
What do your teammates think about your algorithm?

\noindent
\textbf{Memory limit exceeded}

\noindent
What is the max amount of memory your algorithm should need?

\noindent
Are you clearing all data structures between test cases?
